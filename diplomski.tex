% Predlozak za pisanje diplomskog rada na PMF-MO
% Opcenita uputstva za LaTeX se mogu npr. naci na 
% http://web.math.hr/nastava/rp3, http://web.math.hr/nastava/s4-prof/latex.pdf
% NE PREPORUCA se "Ne baš tako kratak uvod u TEX", buduci se radi o vrlo starom prirucniku
% koji nije pogodan za moderne verzije LaTEXa.
% Originalna verzija "The not so short..." na http://tobi.oetiker.ch/lshort/lshort.pdf 
% je obnovljena i daje bolji uvid u moderne verzije LaTeXa

% Stil je optimiziran za kreiranje pdf dokumenta (npr. pomocu pdflatex-a, XeLaTeX-a)

\documentclass[a4paper,oneside,12pt]{memoir} % jednostrano: promijeniti twoside u oneside

% Paket inputenc omogucava direktno unosenje hrvatskih dijakritickih znakova 
% opcija utf8 za unicode (unix, linux, mac)
% opcija cp1250 za windowse
\usepackage[utf8]{inputenc}  % ukoliko se koristi XeLaTeX onda je \usepackage{xunicode}\usepackage{xltxtra}

% Stil za diplomski, unutra je ukljucena podrska za hrvatski jezik
\usepackage{diplomski}
% bibliografija na hrvatskom
\usepackage[languagenames,fixlanguage,croatian]{babelbib} % zahtijeva datoteku croatian.bdf
% hiperlinkovi 
\usepackage[pdftex]{hyperref} % ukoliko se koristi XeLaTeX onda je \usepackage[xetex]{hyperref}

\DisemulatePackage{setspace}
\usepackage{setspace}

% Odabir familije fontova:
% koristenjem XeLaTeX-a mogu se koristiti svi fontovi instalirani na racunalu, npr
% \defaultfontfeatures{Mapping=tex-text}
% \setmainfont[Ligatures={Common}]{Hoefler Text}
% ili
% \newcommand{\nas}[1]{\fontspec{Adobe Garamond Pro}\fontsize{24pt}{24pt}\color{Chocolate}\selectfont #1}
% i onda \nas{Naslov ...}
\usepackage{txfonts} % times new roman 
% Paket graphicx sluzi za manipuliranje grafikom 
\usepackage[pdftex]{graphicx} % ukoliko se koristi XeLaTeX onda je \usepackage[xetex]{graphicx}
% Paket amsmath je vec ukljucen
% Dodatno definirane matematicke okoline:
% teorem (okolina: thm), lema (okolina: lem), korolar (okolina: cor),
% propozicija (okolina: prop), definicija (okolina: defn), napomena (okolina: rem),
% slutnja (okolina: conj), primjer (okolina: exa), dokaz (okolina: proof)
% Definirane su naredbe za ispisivanje skupova N, Z, Q, R i C
% Definirane su naredbe za funkcije koje se u hrvatskoj notaciji oznacavaju drukcije 
% nego u americkoj: tg, ctg, ... (\tgh za tangens hiperbolni)
% Takodjer su definirane naredbe za Ker i Im (da bi se razlikovala od naredbe za imaginarni dio kompleksnog
% broja, naredba se zove \slika).

\pagestyle{headings}
% uz paket fancyhdr mogu se lako kreirati fancy zaglavlja i podnozja



% Podaci koje treba unijeti
\title{DNA kriptografija}
\author{Antonio Kovačić}
\advisor{prof. dr. sc. Andrej Dujella}  % obavezno s titulom (prof. dr. sc ili doc. dr. sc.)
\date{srpanj, 2014.}  % oblika mjesec, godina

% Moguce je unijeti i posvetu
% Ukoliko nema posvete, dovoljno je iskomentirati/izbrisati sljedeci redak 
\dedication{Ovaj diplomski rad posvećujem svojim roditeljima, sestri, braći, prijateljima, mentoru, profesorima, kolegama s faksa, kao i svim ljudima koji su doprinijeli kako mom intelektualnom rastu, tako i mom rastu kao cjelovite osobe.}

\begin{document}

% Naredna frontmatter generira naslovnu stranicu, stranicu za potpise povjerenstva, eventualnu posvetu i sadrzaj
% Moze se iskomentirati ukoliko nije u pitanju konacna verzija
\frontmatter

% Tekst diplomskog ...
\begin{spacing}{1.5}
% Diplomski rad treba poceti s uvodnim poglavljem  
\begin{intro}

U današnje vrijeme svjedoci smo nagloga porasta razmjene podataka. Naglim napretkom današnjih računala, javila se potreba za povećanjem sigurnosti, odnosno zaštite podataka, koji putuju preko komunikacijskog kanala. Današnji kriptosustavi omogućuju siguran prijenos takvih podataka, a ključ njihovog razbijanja zapravo leži u faktoriziranju nekog \textit{velikog} broja (na primjer \textsc{RSA} kriptosustav). Na današnjim računalima, takav problem nije lako riješiv - pa su ti sustavi još uvijek sigurni. Razvojem novih teoretskih modela računala - koji se pokušavaju i u praksi realizirati - uočeno je da problem faktorizacije neće biti više takav problem. Primjer jednog takvog računala je kvantno računalo za kojeg postoji algoritam (\textit{Shorov algoritam}) koji faktorizira broj u polinomnom vremenu.
Time se javila potreba za osmišljanjem  novih teoretskih modela računala - odnosno kriptosustava - koji bi bili otporni na kvantno izračunavanje - ne bi se mogli probiti uporabom kvantnog računala u nekom razumnom vremenu. Takve kriptosustave ćemo zvati \textit{kvantno rezistentnima}.
Tema ovog diplomskog rada biti će DNA kriptografija. Kratko rečeno, radi se o teoretskom modelu kriptografskog sustava koji pomoću DNA izračunavanja šifrira podatke. Prednost takvog sustava jest upravo što je kvantno rezistentan.\\[0.2cm]
U ovom radu najprije ćemo se ukratko upoznati s pojmom DNA računala, odnosno DNA izračunavanja, složenosti DNA računala te algoritmom za enkripciju, odnosno dekripciju podataka pomoću DNA računala.
\end{intro}

\end{spacing}
\end{document}